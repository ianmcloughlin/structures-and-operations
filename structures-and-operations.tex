\documentclass{iansnotes}

\title{Structures and Operations}
\author{ian.mcloughlin@atu.ie}
\date{\today}

\begin{document}

\maketitle

Are all programming languages equally capable?
What are the limits of computation?
How can we discuss computational problems without worrying about the details of a specific machine?

To answer these questions, we need a simple way of describing problems.
In the following article we will define the basic building blocks of computation.


\section{References}
First, we will use Norman Biggs' \emph{Discrete Mathematics}~\autocite{biggs}.
Second, we will use Michael Sipser's \emph{Introduction to the Theory of Computation}~\autocite{sipser}.

\section{Sets} 
  A set is a collection of objects, usually denoted using curly braces\autocite[3]{sipser}.
  For example, the set $A$ below contains the three objects $1$, $2$, and $3$.
  The objects are usually called elements of the set.

  $$ A = \{ 1, 2, 3 \} $$

  Sets can be infinite, in which case the elements can be identified by an algorithm or property.
  In this case we usually assume the infinite set of counting numbers\footnote{The numbers are usually called the natural numbers and $\mathbb{N}_0$ is the set of natural numbers including zero.} $\mathbb{N}_0 = \{ 0, 1, 2, 3, \ldots \}$ is a given\footnote{Sometimes it is convenient to not include the element $0$, in which case we denote the set $\mathbb{N}$.}.

  In the below, the set $T$ of even positive natural numbers is given by an algorithm.
  The algorithm says start with a natural number and multiply it by two.
  The set $P$ is given by the property that each element is prime.

  $$ T = \{ 2n \, | \, n \in \mathbb{N} \} $$
  $$ P = \{ p \in \mathbb{N} \, | \, p \textrm{ is prime} \} $$

  Two important properties of sets are that sets are unordered and that each element is distinct.
  Note there is no mention of order in the definition \emph{collection of objects}\footnote{We can create an order or ordering of a set if we wish but that is something we must treat alongside the set.}.
  Likewise, the idea of an \emph{object} is that it is unique --- we did not say an instance of an object or anything like that.

  We say $B$ is a subset of $A$ if all of the elements on $B$ are also in $A$.
  When $B$ has $k$ elements, we sometimes say $B$ is a $k$-subset of $A$.
  Under this definition, the empty set and $A$ itself are always subsets of a set $A$.

  Note that a set $B$ is an object itself, and so might be an element of a set $A$.
  In this case, we are not saying that the elements of $B$ are individually in $A$, although that could also be the case.
  The distinction is important\footnote{Bertrand Russel is known for Russell's paradox about a set $R$ which is the set of all sets that do not contain themselves. A set seemingly may contain itself --- consider the set of all sets. Does $R$ contain itself?}.


\section{Tuple Notation}
  When order matters, we use tuples.
  Tuples are basically the same as lists of arrays in programming languages.

  A tuple is a finite sequence\autocite[6]{sipser}.
  A sequence is a list of objects, usually stipulated to come from a set or sets.
  A tuple of length $k$ is sometimes called a $k$-tuple, although a $2$-tuple is usually just called a pair.

  The word \emph{list} implies an order --- we can talk about the first thing on a list, if it exists.




\end{document} 